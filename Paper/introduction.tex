Reversible computing is a field that is rapidly gaining interest among researchers due 
to the potential to circumvent the fundamental limitations in energy efficiency and heat 
loss that are being faced by traditional irreversible computing designs. It has been 
suggested that within 20 years we will no longer be able to achieve further increases in 
performance or efficiency at a chip level in irreversible circuits \verb!(Frank, 2005)! It has 
also been long known that reversible logic can lead to circuits with much lower power 
dissipation \verb!(R. Landauer, 1961)! and in theory it is possible to design a reversible 
circuit capable of dissipating zero energy \verb!(Bennett 1973)!. Interest in reversible logic 
is also growing because it has been shown to be applicable to other fields such as 
optical computing \verb!(P.Picton, 1991)!, low power CMOS design \verb!(W.C. Athas & L.J. Svensson, 1994)!, 
nanotechnology \verb!(R.C. Merkle, 1993)!, and quantum computing \verb!(A.N Al-Rabadi, 2004)!.