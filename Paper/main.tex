\documentclass[12pt]{article}
\usepackage[utf8]{inputenc}
\usepackage{amsmath}
\usepackage{amssymb}
\usepackage{amsthm}
\usepackage{float}
\usepackage{fullpage}
\usepackage{microtype}
\usepackage{natbib}
\usepackage{url}
\usepackage{graphicx}
\input{Qcircuit}

\newtheorem{mydef}{Definition}


\title{A Genetic Approach to Reversible Logic Synthesis}
\author{Rio Lowry, Christopher Rabl, Hardipsinh Rana}
\date{}
\begin{document}
\maketitle

\section*{Abstract}

%This paper explores the use of genetic programming in conjunction with a shared cube representation of reversible cascades to minimize the gate count and quantum cost of arbitrary reversible circuits. Rather than optimizing for either quantum cost or cascade length, we utilize a feature vector approach to minimize both aspects. A recent approach by Nayeem and Rice provides a foundation for the shared-cube approach to reversible logic synthesis. Due to the complexity of these operations, we expand on this foundation, combining it with a genetic algorithm to yield a solution. This approach allows us to achieve a solution more quickly than by performing a brute-force search over the problem space.
%We demonstrate the use of a shared cube representation from \cite{Nayeem2011} and apply an evolutionary algorithm to find optimized representations. We consider an implementation that uses a subset of the ordering rules in \cite{Rice2011} as a guideline for implementing our mutation function. We present an implementation of this approach and selected examples using the Python programming language, and compare our results to those achieved in the literature as applied to Revlib benchmarks \cite{RevLib}.

This paper explores the use of genetic programming applied to a representation of reversible cascades in order to reduce the gate count and quantum cost of arbitrary reversible circuits. We present a new software suite for simulating reversible circuits implemented using the Python programming language and discuss methods of parallelization. We present an implementation of this approach, selected results, and a comparison of our results against those achieved in the literature as well as the RevLib benchmarks of \cite{RevLib}.

\begingroup
\let\clearpage\relax
\section{Introduction}

Reversible computing is a field that is rapidly gaining interest among researchers due to its potential to circumvent the fundamental limitations in energy efficiency and heat loss that are faced by traditional irreversible computing designs. It has been suggested that within 20 years, we will no longer be able to achieve further increases in performance or efficiency at a chip level using irreversible circuits (\cite{Frank2005}). It has also been long known that reversible logic can lead to circuits with much lower power dissipation (\cite{Landauer61}) and in theory, it is possible to design a reversible circuit capable of dissipating zero energy (\cite{Bennett73}). Interest in reversible logic is also growing because it has been shown to be applicable to other fields such as optical computing (\cite{Picton91}), low power CMOS design (\cite{Athas94}), nanotechnology (\cite{Merkle1993}), and quantum computing (\cite{AlRabadi2004}).

\section{Background}
\subsection{Unitary Matrices}
Every quantum gate can be represented by a unitary transformation (in the form of a unitary matrix) whose entries are complex variables corresponding to the complex coefficients of a given particle's wave function. Unitary transformations allow us to perform actual computations with qubits since they can be realized using technologies like NMR, for instance: a qubit in an NMR machine undergoes quantum state changes due to a changing magnetic field. These magnetic field changes are, in turn, represented by unitary matrices (\cite{Lukac2003}). 

\begin{mydef}
 A {\bf unitary matrix} is an $n \times n$ matrix of complex coefficients which, when multiplied by its Hermitian, gives the identity matrix. Thus, for a unitary matrix $U$, it is true that $(U^{T})^{*} = U^{-1}$ where $(\cdot)^{*}$ denotes complex conjugation.
\end{mydef}

In order to create useful operations out of ``quantum primitives'', we can compose unitary transformations in order to come up with a permutation representation of a gate or cascade of gates. We can use the ``Square-Root-of-NOT'' gate to construct a NOT gate, for instance:
\[ \sqrt{\text{NOT}} = \frac{1+i}{2}
  \left[
  \begin{matrix}
   1 & -i \\
   -i & 1
  \end{matrix}
  \right] \Rightarrow
  \sqrt{\text{NOT}}*\sqrt{\text{NOT}} = \left(\frac{1+i}{2}\right)^{2} \left[
  \begin{matrix}
   1 & -i \\
   -i & 1
  \end{matrix}
  \right] *
  \left[
  \begin{matrix}
   1 & -i \\
   -i & 1
  \end{matrix}
  \right] =
  \left[
  \begin{matrix}
   0 & 1 \\
   1 & 0
  \end{matrix}
  \right]
\]

There are many other unitary matrices: more examples may be found in \cite{Lukac2003}. 


\subsection{Permutation Matrices}
\begin{mydef}
 A {\bf permutation matrix} $P$ is an $n \times n$ matrix created by permuting the rows of the identity matrix $I_{n}$. It is the case that $P*P^{T}=P^{T}*P=I$ and that $\text{det}(P)=1$. 
\end{mydef}

Rather than deal with unwieldy unitary transformations all the time, we can use permutation matrices, as described by \cite{Williams1999}. These are a powerful tool, since an $n \times n$ permutation matrix can be used to represent a $2^{n} \times 2^{n}$ unitary operation. Additionally, since permutation matrices are sparse, they can be computed with and stored more efficiently than full matrices. As an aside, the unitary matrices of some gates (such as NOT) are also permutation matrices, but the gates which are ``true quantum primitives'', as described in \cite{Lukac2003} have only unitary representations. \\

In brief, permutation matrices encode the rows of a circuit or gate's truth table. Given the truth table for a CNOT gate, for instance, it is quite simple to construct its permutation matrix: we begin by encoding the inputs and outputs of the gate as decimal numbers, and create a mapping $M : \mathbb{Z}_k \rightarrow \mathbb{Z}_k$ between them, where $k=2^{w}$ where $w$ is the ``width'' of the gate, or the number of inputs. Then, we use this mapping to construct a permutation matrix, using the following rule:
\begin{align*}
P = [p_{ij}] \text{ where } p_{ij} = 
  \begin{cases}
   1 & \text{if } i=n \text{ and } j=M(n) \hspace{1em} \forall n \in \mathbb{Z}_{k} \\
   0 & \text{otherwise}
  \end{cases}
\end{align*}

In this case, since CNOT has a width of 2, that means $k=2^{2}=4$, in this case. \\

\begin{tabular}{c c c c c}
\begin{tabular}{c | c || c | c}
 a & b & a' & b' \\ \hline
 0 & 0 & 0 & 0 \\
 0 & 1 & 0 & 1 \\
 1 & 0 & 1 & 1 \\
 1 & 1 & 1 & 0
\end{tabular} & $\rightarrow$ &
\begin{tabular}{c | c}
 $n$ & $M(n)$ \\ \hline
 0 & 0 \\
 1 & 1 \\
 2 & 3 \\
 3 & 2
\end{tabular} & $\rightarrow$ &
$P = \left[
    \begin{matrix}
     1 & 0 & 0 & 0 \\
     0 & 1 & 0 & 0 \\
     0 & 0 & 0 & 1 \\
     0 & 0 & 1 & 0
    \end{matrix}
\right]
$

\end{tabular} \\ \\

A useful property of permutation matrices is that they allow us to ``compose'' permutations. In order to do this, we use the following identity: $P_{\sigma \circ \pi} = P_{\pi}*P_{\sigma}$. Note that the order of the matrix multiplication matters, as matrices do not typically commute under multiplication. Having this composition operator makes it easy to represent cascades in a unique way. We can check that two cascades realize the same function if their output permutation matrices are identical. This provides circuit designers with a way to ``equate'' cascades.

\subsection{Quantum Cost}
Since we can represent operations on qubits using unitary transformations (which conveniently correspond to exactly one quantum operation each), we can devise a metric called ``quantum cost'' in order to determine whether the transformations we perform constitute an efficient synthesis of a given operation. In an NMR system, each electromagnetic pulse to which we subject a qubit has a cost: whether it is the amount of energy required to create the pulse, or the risk of the qubit decohering into a useless state (through vibrations, or other environmental perturbations), these factors may be treated as unitless ``cost'' variables which must be taken into account. \\

As quantum cost is a unitless quantity which corresponds directly to the number of unitary operations in a quantum circuit, it is a very useful metric for calculating the efficiency of an implementation of a circuit. In order to determine the quantum cost of a gate or cascade, we need to break it down into ``quantum primitives'' (unitary transformations). For instance, we can break down a 3-input Toffoli gate like so:

{\begin{align*}
 \Qcircuit @C=2em @R=1.5em {
 \lstick{a} & \qw 	& \ctrl{2}  	& \ctrl{1} & \qw & \ctrl{1} & \qw & \lstick{a'} \\
 \lstick{b} & \ctrl{1} 	& \qw		& \targ & \ctrl{1} & \targ & \qw & \lstick{b'} \\
 \lstick{c} & \gate{V_{0}} & \gate{V_{1}}       & \qw & \gate{V^{+}} & \qw & \qw & \lstick{c'}
 }
\end{align*}}

Of course, it is not immediately obvious why this construction gives us a Toffoli gate. Note that the $\sqrt{\text{NOT}}$ gates (and their Hermitian analogs) do not get activated unless their control lines are 1. \\

So, if we pass $a=0$ and $b=0$ through our gate, $c$ remains unchanged, as do $a$ and $b$. If $a=0$ and $b=1$, then the gate that gets applied to $c$ will be $V_{0}*V^{+}=I$, which is the identity, so $c$ will be unchanged. If $a=1$ and $b=0$, then the gate that gets applied to $c$ will be $V_{1}*V^{+}=I$, so $c$ will be unchanged, and finally, if $a=1$ and $b=1$, $c$ will be inverted because the gate that gets applied will be $V_{0}*V_{1}=\text{NOT}$. Thus, a 3-input Toffoli gate may be simulated by at least 5 quantum primitives, and so it has a quantum cost of 5. This result is due to \cite{Smolin1994}.

\subsection{ESOP Cube List Representations}

Any Boolean function can be represented by an exclusive-or sum-of-products (ESOP) expression. This is particularly useful for reversible logic synthesis since there are existing algorithms for converting any ESOP expression into a cascade of Toffoli gates, thus allowing us to generate a reversible circuit from arbitrary Boolean functions. \\ 


In reversible circuit design, ESOP expressions are often written as a cube list. A cube list is an $ n \times m $ matrix, where $m$ is the number of product terms in the ESOP expression, and $n = i + j$, where $i$ is the number of input variables and $j$ is the number of output variables in the expression. Each of the $m$ rows in the matrix are the ``cubes'' that make up the cube list and represent one of the products from the ESOP expression. \\

Each cube in the list takes the general form: $x_{1} x_{2} ... x_{i} f_{1} f_{2} ... f_{j}$, where each of the elements $x_{1} ... x_{i}$ 
represent an input variable and each element $f_{1} ... f_{j}$ represents an output variable. For each cube in the 
cube list, a $1$ is written in cube position $x_{k}$ , $k \in \{1,2, ..., i\}$ if the variable $x_{k}$ is in 
the ESOP product for that row. A $0$ is written if the negation $\bar{x}_{k}$ is present, and a '$-$' is written if $x_{k}$ is not present in the
product term for that cube. For each element $f_{p}$, $p \in \{1,2,...j\}$, a 1 is written if that output variable contains 
the product represented by the input portion of the list and a 0 is written otherwise. See Figure \ref{fig:cubelist}a for 
an example. \\
%\include figures (a) and (b) showing cube-list and circuit.
\begin{figure}[h]
\centering
 \begin{tabular}{l l l}
  \begin{tabular}{lll | lll}
    $x_{1}$ & $x_{2}$ & $x_{3}$ & $f_{1}$ & $f_{2}$ & $f_{3}$ \\
    \hline
    1 & 1 & 1 & 1 & 1 & 1 \\
    1 & 0 & 1 & 1 & 1 & 0 \\
    0 & 1 & $-$ & 1 & 1 & 1 \\
  \end{tabular} 
  & \ \ \ &
  \begin{tabular}{l}
  \Qcircuit @C=1.5em @R=1.0em {
    \lstick{x_{1}} 	 &  \ctrl{1}      &  \ctrl{1}      &  \ctrl{1}      &  \ctrl{1}      &  \ctrl{1}      &  \qw        &  \qw        &  \qw        &  \qw & \lstick{g_{1}} \\
    \lstick{\bar{x}_{1}} &  \qw           &  \qw           &  \qw           &  \qw           &  \qw           &  \ctrl{1}   &  \ctrl{1}   &  \ctrl{1}   &  \qw & \lstick{g_{2}} \\
    \lstick{x_{2}} 	 &  \ctrl{1} \qwx &  \ctrl{1} \qwx &  \ctrl{1} \qwx &  \qw \qwx      &  \qw \qwx      &  \ctrl{1}   &  \ctrl{1}   &  \ctrl{1}   &  \qw & \lstick{g_{3}} \\
    \lstick{\bar{x}_{2}} &  \qw           &  \qw           &  \qw           &  \ctrl{1} \qwx &  \ctrl{1} \qwx &  \qw        &  \qw        &  \qw        &  \qw & \lstick{g_{4}} \\
    \lstick{x_{3}} 	 &  \ctrl{1} \qwx &  \ctrl{1} \qwx &  \ctrl{1} \qwx &  \ctrl{1}      &  \ctrl{1}      &  \qw \qwx   &  \qw \qwx   &  \qw \qwx   &  \qw & \lstick{g_{5}} \\
    \lstick{\bar{x}_{3}} &  \qw           &  \qw           &  \qw           &  \qw           &  \qw           &  \qw \qwx   &  \qw \qwx   &  \qw \qwx   &  \qw & \lstick{g_{6}} \\
    \lstick{0}  	 &  \targ  \qwx   &  \qw \qwx      &  \qw \qwx      &  \targ  \qwx   &  \qw \qwx      &  \targ \qwx &  \qw \qwx   &  \qw \qwx   &  \qw & \lstick{f_{1}} \\
    \lstick{0}  	 &  \qw           &  \targ \qwx    &  \qw \qwx      &  \qw           &  \targ  \qwx   &  \qw        &  \targ \qwx &  \qw \qwx   &  \qw & \lstick{f_{2}} \\
    \lstick{0}   	 &  \qw           &  \qw           &  \targ \qwx    &  \qw           &  \qw           &  \qw        &  \qw        &  \targ \qwx &  \qw & \lstick{f_{3}} 
    }
  \end{tabular} \\
  \ \ \ \ \ \ \ \ \ \ \ \ (a) & \ \ \ & \ \ \ \ \ \ \ \ \ \ \ \ \ \ \ \ \ \ \ \ \ \ \ \ \ \ \ \ \ \ (b)
 \end{tabular}
 \caption{(a) The cube list and (b) resulting circuit.}
  \label{fig:cubelist}
\end{figure}
 
Given an ESOP expression encoded as a cube list, \cite{Thornton2007} proposed a 
method for reversible logic synthesis that implements the function as a reversible circuit using only Toffoli gates. In this method, an empty cascade with $2i + j$ lines is created. Two input lines are given for every input 
variable where one line corresponds to the variable $x_{k}$, the next line corresponds to its negation, $\bar{x}_{k}$. The remaining $j$ lines correspond to the output variables. For every output line $f_{p}$, a Toffoli gate 
is placed with its target on line $f_{p}$ and a control for is placed on the input line $x_{k}$ if there is a 1 in the cube for the corresponding variable, or if we encounter a 0 in the
cube for $x_{k}$, we place the control on the negation line of $x_{k}$. see Figure \ref{fig:cubelist}b. This method allows a cube list to be efficiently transformed into a reversible cascade and allows for the synthesis of large functions. \\*




\subsection{ESOP Cube List Ordering Rules}


Modifications to the above method %with the use of inverters instead of a negation line, careful ordering 
%of cubes to reduce the number of not gates and and sharing cubes between output lines 
have both reduced the number of lines and the number of gates required to implement these circuits. % see \emph{[To add citations for these papers]}.  
In \cite{Nayeem2011}, for example, a number of rules were proposed that manipulate the cube list representation to create a new cube list with 
the same output as the first one, 
depending on the rule and the state of the cube that the rule is being applied to these rules may increase 
or decrease the number of cubes in the list.\\


%\noindent \emph{[To add brief description of ordering rules as these are to be used in our mutation function ...]}

\newcommand{\tab}{\hspace*{2em}}

\subsection{Genetic Algorithm}
 The Genetic Algorithm is a search heuristic that was introduced and investigated by John Holland (1975) and his students (DeJong, 1975). The algorithm attempts to mimic the evolutionary process of natural selection by modelling the concepts of individuals in a population, fitness and selection, crossover, and mutation that are found in the biological reproduction of organisms (\cite{Mitchell1996}). \\

Over a number of generations, fit individuals (effective solutions to a problem) evolve out of the previous generations of individuals in a population. While there exist many variations on this theme, the basic algorithm is as follows:
\begin{enumerate}
 \item Generate an initial population of individuals that represent potential solutions to the problem, typically in a randomized fashion.
 \item Evaluate the ``fitness'' of each individual in the population according to some metric (a fitness function).
 \item Until a solution is found or a maximum number of generations have elapsed, repeat the following process:
  \begin{enumerate}
   \item Select individuals from the population to reproduce. This is typically done according to the heuristic provided by the fitness function.
   \item Perform crossover between individuals in order to create a new population.
   \item Apply mutations to certain members of the population in order to expand the search space.
   \item Evaluate each individual's fitness and keep track of the top individuals since they may be candidates for the next selection.
  \end{enumerate}
\end{enumerate}


\subsubsection{Representation}
Representation of the individuals in a population is critical to the successful application of this algorithm and is perhaps the most challenging aspect of implementing it. Since the model is based on biology, each individual is often 
encoded as a string to which the mutation and crossover functions are applied. \\

To show how the genetic algorithm works, we will use a simple example of adding two numbers to reach a specified sum. Each individual in our algorithm will be represented as bit strings consisting of 6 ``genes'' with each gene composed of 4
bits. Thus, an individual in our example may be represented as: 0001 1110 1001 1111 0110 0000. We choose an
encoding scheme where each gene in the individual encodes either a number or an operator. We will let the genes $0000$ to 
$1101$ represent the numbers 0 through 13 and the genes $1110$ and $1111$ represent addition and subtraction operators, respectively. Using 
this encoding scheme, our individual would represent the arithmetic expression $1 + 9 - 70$.

\subsubsection{Fitness and Selection}
The fitness function allows us to measure how close the solution represented by any individual is to the desired output solution. Each individual is evaluated and ranked according to its fitness, and then according to some selection 
criteria where candidates are chosen for crossover and mutation. To measure the fitness, we need to decode the representation of each 
individual and compare it against our goal result. From the example individual above, if our goal is to find an expression 
whose result is equal to the absolute value of the number $30$, our fitness function would decode the individual's genetic representation and find that 
it has a value of $-60$. The individual would then be compared to the goal and its fitness would be evaluated, yielding a value between $0.0$ and $1.0$, inclusive. Once every individual is ranked according 
to its fitness, candidates are selected for crossover and mutation according to the selection criteria specified.


\subsubsection{Crossover}
Crossover attempts to model the process of sexual reproduction found in nature. After ``parent'' individuals are selected from the population of a generation, the genes of these individuals are combined in order to produce one or more child individuals. To 
combine their genes, a crossover point is specified in each parent. Consider the following two binary strings, $A$ = 1010 1100 0001 1000 1111 0000 
and $B$ = 1011 0011 1010 1111 0000 1110. Their offspring individuals might be a combination of the first part of $A$ with the second part of $B$, 
or the first part of $A$ combined with the second part of $B$. Here is an illustration of crossover using a randomly chosen crossover point, $X$:

\begin{center}
$A$ = 1010 1100 0001 X 1000 1111 0000
\\*$B$ = 1011 0011 1010 X 1111 0000 1110
\vspace{2 mm}
\\*We now have the following fragments: 
\\*$A_{1}$ = 1010 1100 0001 
\\*$A_{2}$ = 1000 1111 0000
\\*$B_{1}$ = 1011 0011 1010 
\\*$B_{2}$ = 1111 0000 1110
\vspace{2 mm}
\\*Combining the fragments of $A$ and $B$ we would yield two offspring, $A_{1}B_{2}$ and $B_{1}A_{2}$:
\vspace{2 mm}
\\*$A_{1}B_{2}$ = 1010 1100 0001 1111 0000 1110
\\*$B_{1}A_{2}$ = 1011 0011 1010 1000 1111 0000
\end{center}

\subsubsection{Mutation}
The mutation operation models the natural mutation of genes in living organisms. For selected individuals, individual genes may be randomly 
changed in some way. Depending on algorithm and encoding, this mutation is usually achived through addition, deletion, or swapping of bits.
For example given the individual $A$ = 1010 1100 0001 1000 1111 0000, mutation could occur by swapping a bit. An individual resulting from this operation might be: 
\begin{center} $A_{sw}$ = 1010 1\underline{0}00 0001 1000 1111 0000 \end{center}
If instead we replace a gene with another gene within the individual, we might obtain:
\begin{center} $A_{re}$ =   1010 \underline{1100} 1110 1000 1111 0000 \end{center}
                                                           
\subsubsection{Initial Parameter Variations}

There are a number of other factors in the design of genetic algorithms that can impact the likelihood of finding a useful (or convergent, in some cases) result. 
These factors include:
\begin{itemize}
 \item {\bf Initial population size:}
\\ Having less individuals in the initial popluation may not allow us to search enough of the search space to find an acceptable solution. Conversely, having too many may result in the algorithm searching more of the search space than necessary.
 \item {\bf Maximum number of generations:}
\\ Specifying a maximum number of generations ensures that search will not continue indefinitely. However, if this number is too small it may limit the algorithm's ability to find a reasonable solution.
 \item {\bf Fitness threshold:}
\\Implementing a fitness threshold allows us to specify a desired ``precision'' for the solution. Depending on the application, an acceptable solution could be one that is very close to the ideal solution but does not exactly provide the
desired result, whereas in other situations, an acceptable solution could be the unique solution that provides a desired result. 
\end{itemize}
 

\subsubsection{Applications of the Genetic Algorithm to Reversible Logic Synthesis}

There have been a number of applications of the genetic algorithm to logic synthesis in general, and reversible logic synthesis in 
particular. \cite{Lukac2003} used genetic algorithm for reversible circuits to generate near-optimal circuits, to which they subsequently applied 
optimization transformations to generate optimal circuits. \cite{Lukac2008} implemented symbolic synthesis within the genetic algorithm to reduce the complexity of the resulting circuits, 
\cite{Khan2004} described a new genetic algorithm-based synthesis method for ternary quantum circuits which reduced gate count in some instances, and 
\cite{Aguirre2003} proposed the use of information theory as the basis for designing a fitness function for Boolean circuit design using genetic programming.

\subsection{Parallelization}

Parallelization is an approach to computational problem solving where the computation is divided into smaller 
subproblems and each sub-problem is computed simultaneously. The results of each sub-problem are then combined 
to get the final result of the whole computation.

The process of parallelizing a computation can be taken at different levels, from the bit level on a single machine 
to distributed computing over multiple machines (using cluster or grid computing).

\paragraph{multiple threading and processes}
Independent computation tasks may be delegated across separate processor cores by using threads or processes. When processing large cascades, we can make use of these techniques in order to reduce computation time.

{\bf Avoiding Global Interpreter Lock} When using an interpreted programming language such as Python, it is important to keep in mind that if each thread is running in the same interpreter instance, it is possible that one thread may ``lock'' the interpreter, preventing the execution of the other threads. Thus, rather than using threads, Revsim uses \emph{subprocesses} in order to delegate tasks to separate processor cores. Each subprocess runs its own interpreter instance, thus sidestepping Global Interpreter Lock. This advantage comes at the cost of increased interpreter overhead, but this cost is negligible when the benefits of subprocessing are considered.

\paragraph{Grid Computing}
add basic definition: which is each node is generally loosely coupled, generally heterogeneous.

\section{Our Approach}

%\emph{This section will detail our approach to the genetic algorithm, 
%including the planned use of the cube reordering rules in our mutation function
%and a general description of our sofware and the hardware it runs on}

\subsection{Revsim: Reversible Logic Simulator}
  \subsubsection{Overview}

Representing reversible circuits in a way that we could both easily simulate them and process them through 
a genetic algorithm was one of the first challenges we faced. There have been a number of reversible approaches 
to circuit synthesis using the genetic algorithm \emph(references lukac, etc) but after conducting a preliminary 
review, none of them offered the flexibility and extensibility that we were seeking. We set out to develop a 
software suite that could would allow us to simulate any number of reversible circuits and let us manipulate those 
circuits at the gate level. We will provide a brief overview and description of the development of our software below.

  \subsubsection{Explanation of Simulator Design}

We first developed an initial version of our circuit simulator using a functional programming approach but once we had conducted some initial 
experimentation and testing we decided to refactor our code using an object-oriented approach.

Figure 3 shows the basic class structure of our software. Conceptually a gate has a number of input lines, output lines, 
controls and targets. The abstract Gate class formalizes this basic representation of a gate and is subclassed to create 
the various types of gates and their implementations.

\begin{figure}[ht!]
\centering
\includegraphics[width=140mm]{diagrams/architecture.png}
\caption{Class structure of Revsim}
\label{overflow}
\end{figure}

Figure 4 shows there are three main types of gates that we have represented in our simulator: Single Target 
gates, such as the the Toffoli, Multiple Target gates, like the Fredkin or Swap, and Same Target gates such as the inverter 
where the target and the control are on the same line.

\begin{figure}[ht!]
\centering
\includegraphics[width=100mm]{diagrams/gate_inheritence.png}
\caption{Gate inheritence in Revsim}
\label{overflow}
\end{figure}
The Cascade class is used to represent a reversible cascade of gates. It provides the primary functionality for modelling 
and modifying our reversible circuits, and is used other classes such as the Truth Table and Genetic Algorithm classes as 
described below.

The Truth Table class allows us to generate the truth tables of our reversible cascade and compare all or part of the truth 
table from one circuit with another. Since we have to propagate values across each gate in the cascade to generate the output 
values and need to generate all \(2^{n}\) entries in the truth table, the process of generating the truth table is linear in the 
number of gates of the circuit is exponential in the number of variables 

Our simulation software also has a number of classes that provide additional extensibility. For example there are input/output 
classes that allow us to directly read from and write to files in the \url{www.revlib.org}'s .real file format. Our initial Genetic Algorithm implementation 
required us to manually enter each goal cascade that we wanted to test against, but using these helper classes we were able to 
implement the ability to parse any .real file and use its target output function in the fitness calculations for our Genetic 
Algorithm. We also developed an experimental Shared Cube class that allows us to generate and use the shared cube 
representation that we were initially going to use to represent individuals in our genetic algorithm so that we could apply a 
number of the rule based transformations from (ref: jackies paper on rule transformations) but after implementing it we decided 
on using the representation described below. 

  \subsubsection{Description of Our Genetic Algorithm}

\paragraph{Representation.} 

We initially explored using a shared cube representation for the individuals in used in our algorithm however while we initially 
thought that the transformational rules  in [reference] would be useful in implementing mutation we found the difficult to implement 
in the context of our genetic algorithm and we not able to discover an appropriate method of implementing crossover using the shared 
cube representation. Instead we settled on using the cascade representation from our Cascade class which stores the list of gates used 
in the circuit for the individuals in our populations.

Our algorithm initially reads in a circuit that specifies the desired output behaviour and then creates the initial population as copies 
of the initial circuit that have been mutated from the initial circuit up to a maximum number of mutations specified in the initial 
conditions.once the initial population is generated, the cycle of fitness evaluations, selection mutation and crossover repeat until 
one of two terminal conditions are reached, either a fitness of 1.0 or a maximum number of generations.

\paragraph{Fitness and Selection.} 

Our fitness metric measured how closely the truth table of the current individual matched the truth table of the of 
the desired output behaviour. It basically performs an exhaustive comparison and so the number of comparisons needed are approximately
 exponential in the number of variables (linewidth) of the circuit.  


\paragraph{Crossover.}

Our implementation of crossover selects the best two individuals as parents and creates two child individuals. The first child has the 
first half of the gates from parent 1 and the second half from parent 2 while the second child has the first half of the gates from 
parent 2 and the second half from parent 1, The children are then added to the population for the next iteration of the algorithm. 

\paragraph{Mutation.}

The mutation function randomly selects a certain number of gates and will either replace or remove them from the cascade of the 
individuals it it being applied to.

\paragraph{Parameter Variations.}

Our initial parameters


\subsubsection{Parallelization}

We found that with larger circuits processing time became significantly longer so we looked at a few ways to parallelize our simulations 
in order to gather a larger dataset of results in a shorter period of time.

\paragraph{Grid Computing.}

We first spent some time adapting our software so that we could run it across the university of Lethbridge's HTcondor computing grid. 
Because we needed to test a number of circuits across a set of variable initial parameters one of the key challenges we faced was 
automating the task of creating the HTcondor job submissions. We were able to script a solution that allows us to automatically take 
a batch of any number reversible circuits in .real format and output a set of submission-ready executables that we could run across 
the HTcondor grid. This provided a significant increase in the speed we were able to generate tests and obtain results.

\paragraph{Parallelization of large circuits.}

Even using condor we still faced the problem that that the processing time for runs of our genetic algorithm with large circuits was 
significantly longer. In order to mitigate this we began experimenting with spitting the processing of each individual circuit into 
multiple processes that we can run concurrently on each of the processors of the machine running the job. Preliminary tests have been 
promising.   \emph{  DOUBLE CHECK THIS STATEMENT!!! is is really promising??}

\section{Experimental Results}

\begin{table}
    \begin{tabular}{l | l | l | l | l}
    Circuit & Data       & Number of & \verb!%! Improved & Best      \\
            & Generating & Improved  &                   & Reported   \\
            & Runs       & Circuits  &                   & Improvement \\ \hline
    \verb!Max46_240!     & 5000                 & 5000                        & 100.00    & 14038                     \\
    \verb!life_238!      & 4999                 & 4999                        & 100.00    & 11084                     \\
    \verb!f51m_233!    & 60                   & 60                          & 100.00    & 4472                      \\
    \verb!alu-v0_27!     & 1020                 & 904                         & 88.63     & 7                         \\
    \verb!decod24-v3_45! & 1020                 & 838                         & 82.16     & 31                        \\
    \verb!gsym9!         & 1040                 & 665                         & 63.94     & 52                        \\
    \verb!4gt11_83!      & 1020                 & 576                         & 56.47     & 9                         \\
    \verb!gsym6!         & 1020                 & 536                         & 52.55     & 6                         \\
    \verb!rd32_273!      & 1020                 & 526                         & 51.57     & 24                        \\
    \verb!mod5adder_129! & 1277                 & 188                         & 14.72     & 72                        \\
    \verb!hwb7_62!       & 60                   & 7                           & 11.67     & 217                       \\
    \verb!urf4_187!      & 8                    & 0                           & 0.00      & N/A                       \\
    \verb!hwb8_116!      & 14                   & 1                           & 7.14      & 100                       \\
    \verb!seq_314!      & 0                    & 0                           & N/A       & N/A                       \\
    \verb!hwb9_121!      & 0                    & 0                           & N/A       & N/A                       \\
    \end{tabular}
\end{table}




\section{Conclusion}

\emph{This section will detail our conclusion}
\endgroup

\pagebreak
\bibliographystyle{plainnat}
\bibliography{bibliography}

\end{document}
