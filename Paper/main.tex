\documentclass[12pt]{article}
\usepackage[utf8]{inputenc}
\usepackage{amsmath}
\usepackage{amssymb}
\usepackage{amsthm}
\usepackage{fullpage}
\usepackage{microtype}
\usepackage{natbib}
\usepackage{url}
%    Q-circuit version 2
%    Copyright (C) 2004  Steve Flammia & Bryan Eastin
%    Last modified on: 9/16/2011
%
%    This program is free software; you can redistribute it and/or modify
%    it under the terms of the GNU General Public License as published by
%    the Free Software Foundation; either version 2 of the License, or
%    (at your option) any later version.
%
%    This program is distributed in the hope that it will be useful,
%    but WITHOUT ANY WARRANTY; without even the implied warranty of
%    MERCHANTABILITY or FITNESS FOR A PARTICULAR PURPOSE.  See the
%    GNU General Public License for more details.
%
%    You should have received a copy of the GNU General Public License
%    along with this program; if not, write to the Free Software
%    Foundation, Inc., 59 Temple Place, Suite 330, Boston, MA  02111-1307  USA

% Thanks to the Xy-pic guys, Kristoffer H Rose, Ross Moore, and Daniel Müllner,
% for their help in making Qcircuit work with Xy-pic version 3.8.  
% Thanks also to Dave Clader, Andrew Childs, Rafael Possignolo, Tyson Williams,
% Sergio Boixo, Cris Moore, Jonas Anderson, and Stephan Mertens for helping us test 
% and/or develop the new version.

\usepackage{xy}
\xyoption{matrix}
\xyoption{frame}
\xyoption{arrow}
\xyoption{arc}

\usepackage{ifpdf}
\ifpdf
\else
\PackageWarningNoLine{Qcircuit}{Qcircuit is loading in Postscript mode.  The Xy-pic options ps and dvips will be loaded.  If you wish to use other Postscript drivers for Xy-pic, you must modify the code in Qcircuit.tex}
%    The following options load the drivers most commonly required to
%    get proper Postscript output from Xy-pic.  Should these fail to work,
%    try replacing the following two lines with some of the other options
%    given in the Xy-pic reference manual.
\xyoption{ps}
\xyoption{dvips}
\fi

% The following resets Xy-pic matrix alignment to the pre-3.8 default, as
% required by Qcircuit.
\entrymodifiers={!C\entrybox}

\newcommand{\bra}[1]{{\left\langle{#1}\right\vert}}
\newcommand{\ket}[1]{{\left\vert{#1}\right\rangle}}
    % Defines Dirac notation. %7/5/07 added extra braces so that the commands will work in subscripts.
\newcommand{\qw}[1][-1]{\ar @{-} [0,#1]}
    % Defines a wire that connects horizontally.  By default it connects to the object on the left of the current object.
    % WARNING: Wire commands must appear after the gate in any given entry.
\newcommand{\qwx}[1][-1]{\ar @{-} [#1,0]}
    % Defines a wire that connects vertically.  By default it connects to the object above the current object.
    % WARNING: Wire commands must appear after the gate in any given entry.
\newcommand{\cw}[1][-1]{\ar @{=} [0,#1]}
    % Defines a classical wire that connects horizontally.  By default it connects to the object on the left of the current object.
    % WARNING: Wire commands must appear after the gate in any given entry.
\newcommand{\cwx}[1][-1]{\ar @{=} [#1,0]}
    % Defines a classical wire that connects vertically.  By default it connects to the object above the current object.
    % WARNING: Wire commands must appear after the gate in any given entry.
\newcommand{\gate}[1]{*+<.6em>{#1} \POS ="i","i"+UR;"i"+UL **\dir{-};"i"+DL **\dir{-};"i"+DR **\dir{-};"i"+UR **\dir{-},"i" \qw}
    % Boxes the argument, making a gate.
\newcommand{\meter}{*=<1.8em,1.4em>{\xy ="j","j"-<.778em,.322em>;{"j"+<.778em,-.322em> \ellipse ur,_{}},"j"-<0em,.4em>;p+<.5em,.9em> **\dir{-},"j"+<2.2em,2.2em>*{},"j"-<2.2em,2.2em>*{} \endxy} \POS ="i","i"+UR;"i"+UL **\dir{-};"i"+DL **\dir{-};"i"+DR **\dir{-};"i"+UR **\dir{-},"i" \qw}
    % Inserts a measurement meter.
    % In case you're wondering, the constants .778em and .322em specify
    % one quarter of a circle with radius 1.1em.
    % The points added at + and - <2.2em,2.2em> are there to strech the
    % canvas, ensuring that the size is unaffected by erratic spacing issues
    % with the arc.
\newcommand{\measure}[1]{*+[F-:<.9em>]{#1} \qw}
    % Inserts a measurement bubble with user defined text.
\newcommand{\measuretab}[1]{*{\xy*+<.6em>{#1}="e";"e"+UL;"e"+UR **\dir{-};"e"+DR **\dir{-};"e"+DL **\dir{-};"e"+LC-<.5em,0em> **\dir{-};"e"+UL **\dir{-} \endxy} \qw}
    % Inserts a measurement tab with user defined text.
\newcommand{\measureD}[1]{*{\xy*+=<0em,.1em>{#1}="e";"e"+UR+<0em,.25em>;"e"+UL+<-.5em,.25em> **\dir{-};"e"+DL+<-.5em,-.25em> **\dir{-};"e"+DR+<0em,-.25em> **\dir{-};{"e"+UR+<0em,.25em>\ellipse^{}};"e"+C:,+(0,1)*{} \endxy} \qw}
    % Inserts a D-shaped measurement gate with user defined text.
\newcommand{\multimeasure}[2]{*+<1em,.9em>{\hphantom{#2}} \qw \POS[0,0].[#1,0];p !C *{#2},p \drop\frm<.9em>{-}}
    % Draws a multiple qubit measurement bubble starting at the current position and spanning #1 additional gates below.
    % #2 gives the label for the gate.
    % You must use an argument of the same width as #2 in \ghost for the wires to connect properly on the lower lines.
\newcommand{\multimeasureD}[2]{*+<1em,.9em>{\hphantom{#2}} \POS [0,0]="i",[0,0].[#1,0]="e",!C *{#2},"e"+UR-<.8em,0em>;"e"+UL **\dir{-};"e"+DL **\dir{-};"e"+DR+<-.8em,0em> **\dir{-};{"e"+DR+<0em,.8em>\ellipse^{}};"e"+UR+<0em,-.8em> **\dir{-};{"e"+UR-<.8em,0em>\ellipse^{}},"i" \qw}
    % Draws a multiple qubit D-shaped measurement gate starting at the current position and spanning #1 additional gates below.
    % #2 gives the label for the gate.
    % You must use an argument of the same width as #2 in \ghost for the wires to connect properly on the lower lines.
\newcommand{\control}{*!<0em,.025em>-=-<.2em>{\bullet}}
    % Inserts an unconnected control.
\newcommand{\controlo}{*+<.01em>{\xy -<.095em>*\xycircle<.19em>{} \endxy}}
    % Inserts a unconnected control-on-0.
\newcommand{\ctrl}[1]{\control \qwx[#1] \qw}
    % Inserts a control and connects it to the object #1 wires below.
\newcommand{\ctrlo}[1]{\controlo \qwx[#1] \qw}
    % Inserts a control-on-0 and connects it to the object #1 wires below.
\newcommand{\targ}{*+<.02em,.02em>{\xy ="i","i"-<.39em,0em>;"i"+<.39em,0em> **\dir{-}, "i"-<0em,.39em>;"i"+<0em,.39em> **\dir{-},"i"*\xycircle<.4em>{} \endxy} \qw}
    % Inserts a CNOT target.
\newcommand{\qswap}{*=<0em>{\times} \qw}
    % Inserts half a swap gate.
    % Must be connected to the other swap with \qwx.
\newcommand{\multigate}[2]{*+<1em,.9em>{\hphantom{#2}} \POS [0,0]="i",[0,0].[#1,0]="e",!C *{#2},"e"+UR;"e"+UL **\dir{-};"e"+DL **\dir{-};"e"+DR **\dir{-};"e"+UR **\dir{-},"i" \qw}
    % Draws a multiple qubit gate starting at the current position and spanning #1 additional gates below.
    % #2 gives the label for the gate.
    % You must use an argument of the same width as #2 in \ghost for the wires to connect properly on the lower lines.
\newcommand{\ghost}[1]{*+<1em,.9em>{\hphantom{#1}} \qw}
    % Leaves space for \multigate on wires other than the one on which \multigate appears.  Without this command wires will cross your gate.
    % #1 should match the second argument in the corresponding \multigate.
\newcommand{\push}[1]{*{#1}}
    % Inserts #1, overriding the default that causes entries to have zero size.  This command takes the place of a gate.
    % Like a gate, it must precede any wire commands.
    % \push is useful for forcing columns apart.
    % NOTE: It might be useful to know that a gate is about 1.3 times the height of its contents.  I.e. \gate{M} is 1.3em tall.
    % WARNING: \push must appear before any wire commands and may not appear in an entry with a gate or label.
\newcommand{\gategroup}[6]{\POS"#1,#2"."#3,#2"."#1,#4"."#3,#4"!C*+<#5>\frm{#6}}
    % Constructs a box or bracket enclosing the square block spanning rows #1-#3 and columns=#2-#4.
    % The block is given a margin #5/2, so #5 should be a valid length.
    % #6 can take the following arguments -- or . or _\} or ^\} or \{ or \} or _) or ^) or ( or ) where the first two options yield dashed and
    % dotted boxes respectively, and the last eight options yield bottom, top, left, and right braces of the curly or normal variety.  See the Xy-pic reference manual for more options.
    % \gategroup can appear at the end of any gate entry, but it's good form to pick either the last entry or one of the corner gates.
    % BUG: \gategroup uses the four corner gates to determine the size of the bounding box.  Other gates may stick out of that box.  See \prop.

\newcommand{\rstick}[1]{*!L!<-.5em,0em>=<0em>{#1}}
    % Centers the left side of #1 in the cell.  Intended for lining up wire labels.  Note that non-gates have default size zero.
\newcommand{\lstick}[1]{*!R!<.5em,0em>=<0em>{#1}}
    % Centers the right side of #1 in the cell.  Intended for lining up wire labels.  Note that non-gates have default size zero.
\newcommand{\ustick}[1]{*!D!<0em,-.5em>=<0em>{#1}}
    % Centers the bottom of #1 in the cell.  Intended for lining up wire labels.  Note that non-gates have default size zero.
\newcommand{\dstick}[1]{*!U!<0em,.5em>=<0em>{#1}}
    % Centers the top of #1 in the cell.  Intended for lining up wire labels.  Note that non-gates have default size zero.
\newcommand{\Qcircuit}{\xymatrix @*=<0em>}
    % Defines \Qcircuit as an \xymatrix with entries of default size 0em.
\newcommand{\link}[2]{\ar @{-} [#1,#2]}
    % Draws a wire or connecting line to the element #1 rows down and #2 columns forward.
\newcommand{\pureghost}[1]{*+<1em,.9em>{\hphantom{#1}}}
    % Same as \ghost except it omits the wire leading to the left. 


\newtheorem{mydef}{Definition}


\title{CPSC 4210 - Final Paper}
\author{R. Lowry, C. Rabl, R. Rana}
\date{}
\begin{document}
\maketitle

\section*{Abstract}

his paper explores the use of genetic programming in conjunction with a shared cube representation of reversible cascades to minimize the gate count and quantum cost of arbitrary reversible circuits. Rather than optimizing for either quantum cost or cascade length, we utilize a feature vector approach to minimize both aspects. A recent approach by Nayeem and Rice provides a foundation for the shared-cube approach to reversible logic synthesis. Due to the complexity of these operations, we expand on this foundation, combining it with a genetic algorithm to yield a solution. This approach allows us to achieve a solution more quickly than by performing a brute-force search over the problem space.
We demonstrate the use of a shared cube representation from \cite{Nayeem2011} and apply an evolutionary algorithm to find optimized representations. We consider an implementation that uses a subset of the ordering rules in \cite{Rice2011} as a guideline for implementing our mutation function. We present an implementation of this approach and selected examples using the Python programming language, and compare our results to those achieved in the literature as applied to Revlib benchmarks \cite{RevLib}.

Reversible computing is a field that is rapidly gaining interest among researchers due 
to the potential to circumvent the fundamental limitations in energy efficiency and heat 
loss that are being faced by traditional irreversible computing designs. It has been 
suggested that within 20 years we will no longer be able to achieve further increases in 
performance or efficiency at a chip level in irreversible circuits \cite{Frank2005} It has 
also been long known that reversible logic can lead to circuits with much lower power 
dissipation \cite{Landauer61} and in theory it is possible to design a reversible 
circuit capable of dissipating zero energy \cite{Bennett73}. Interest in reversible logic 
is also growing because it has been shown to be applicable to other fields such as 
optical computing \cite{Picton91}, low power CMOS design \cite{Athas94}, 
nanotechnology \cite{Merkle1993}, and quantum computing \cite{AlRabadi2004}.

\subsection{ESOP cube-list Representations}

Any Boolean function can be represented by an exclusive-or sum-of-products (ESOP) expression. This is particularly useful for reversible logic synthesis since there are existing algorithms for converting any ESOP expression into a cascade of Toffoli gates, thus allowing us to generate a reversible circuit from arbitrary Boolean functions. \\ 


In reversible circuit design, ESOP expressions are often written as a cube list. A cube list is an $ n \times m $ matrix, where $m$ is the number of product terms in the ESOP expression, and $n = i + j$, where $i$ is the number of input variables and $j$ is the number of output variables in the expression. Each of the $m$ rows in the matrix are the ``cubes'' that make up the cube list and represent one of the products from the ESOP expression. \\

Each cube in the list takes the general form: $x_{1} x_{2} ... x_{i} f_{1} f_{2} ... f_{j}$, where each of the elements $x_{1} ... x_{i}$ 
represent an input variable and each element $f_{1} ... f_{j}$ represents an output variable. For each cube in the 
cube list, a $1$ is written in cube position $x_{k}$ , $k \in \{1,2, ..., i\}$ if the variable $x_{k}$ is in 
the ESOP product for that row. A $0$ is written if the negation $\bar{x}_{k}$ is present, and a '$-$' is written if $x_{k}$ is not present in the
product term for that cube. For each element $f_{p}$, $p \in \{1,2,...j\}$, a 1 is written if that output variable contains 
the product represented by the input portion of the list and a 0 is written otherwise. See Figure \ref{fig:cubelist}a for 
an example. \\
%\include figures (a) and (b) showing cube-list and circuit.
\begin{figure}[h]
\centering
 \begin{tabular}{l l l}
  \begin{tabular}{lll lll}
    $x_{1}$ & $x_{2}$ & $x_{3}$ & $f_{1}$ & $f_{2}$ & $f_{3}$ \\
    \hline
    1 & 1 & 1 & 1 & 1 & 1 \\
    1 & 0 & 1 & 1 & 1 & 0 \\
    0 & 1 & $-$ & 1 & 1 & 1 \\
  \end{tabular} 
  & \ \ \ &
  \begin{tabular}{l}
  \Qcircuit @C=1.5em @R=1.0em {
    \lstick{x_{1}} 	 &  \ctrl{1}      &  \ctrl{1}      &  \ctrl{1}      &  \ctrl{1}      &  \ctrl{1}      &  \qw        &  \qw        &  \qw        &  \qw & \lstick{g_{1}} \\
    \lstick{\bar{x}_{1}} &  \qw           &  \qw           &  \qw           &  \qw           &  \qw           &  \ctrl{1}   &  \ctrl{1}   &  \ctrl{1}   &  \qw & \lstick{g_{2}} \\
    \lstick{x_{2}} 	 &  \ctrl{1} \qwx &  \ctrl{1} \qwx &  \ctrl{1} \qwx &  \qw \qwx      &  \qw \qwx      &  \ctrl{1}   &  \ctrl{1}   &  \ctrl{1}   &  \qw & \lstick{g_{3}} \\
    \lstick{\bar{x}_{2}} &  \qw           &  \qw           &  \qw           &  \ctrl{1} \qwx &  \ctrl{1} \qwx &  \qw        &  \qw        &  \qw        &  \qw & \lstick{g_{4}} \\
    \lstick{x_{3}} 	 &  \ctrl{1} \qwx &  \ctrl{1} \qwx &  \ctrl{1} \qwx &  \ctrl{1}      &  \ctrl{1}      &  \qw \qwx   &  \qw \qwx   &  \qw \qwx   &  \qw & \lstick{g_{5}} \\
    \lstick{\bar{x}_{3}} &  \qw           &  \qw           &  \qw           &  \qw           &  \qw           &  \qw \qwx   &  \qw \qwx   &  \qw \qwx   &  \qw & \lstick{g_{6}} \\
    \lstick{0}  	 &  \targ  \qwx   &  \qw \qwx      &  \qw \qwx      &  \targ  \qwx   &  \qw \qwx      &  \targ \qwx &  \qw \qwx   &  \qw \qwx   &  \qw & \lstick{f_{1}} \\
    \lstick{0}  	 &  \qw           &  \targ \qwx    &  \qw \qwx      &  \qw           &  \targ  \qwx   &  \qw        &  \targ \qwx &  \qw \qwx   &  \qw & \lstick{f_{2}} \\
    \lstick{0}   	 &  \qw           &  \qw           &  \targ \qwx    &  \qw           &  \qw           &  \qw        &  \qw        &  \targ \qwx &  \qw & \lstick{f_{3}} 
    }
  \end{tabular} \\
  \ \ \ \ \ \ \ \ \ \ \ \ (a) & \ \ \ & \ \ \ \ \ \ \ \ \ \ \ \ \ \ \ \ \ \ \ \ \ \ \ \ \ \ \ \ \ \ (b)
 \end{tabular}
 \caption{(a) The cube list and (b) resulting circuit.}
  \label{fig:cubelist}
\end{figure}
 
Given an ESOP expression encoded as a cube list, \cite{Thornton2007} proposed a 
method for reversible logic synthesis that implements the function as a reversible circuit using only Toffoli gates. In this method, an empty cascade with $2i + j$ lines is created. Two input lines are given for every input 
variable where one line corresponds to the variable $x_{k}$, the next line corresponds to its negation, $\bar{x}_{k}$. The remaining $j$ lines correspond to the output variables. For every output line $f_{p}$, a Toffoli gate 
is placed with its target on line $f_{p}$ and a control for is placed on the input line $x_{k}$ if there is a 1 in the cube for the corresponding variable, or if we encounter a 0 in the
cube for $x_{k}$, we place the control on the negation line of $x_{k}$. see Figure \ref{fig:cubelist}b. This method allows a cube list to be efficiently transformed into a reversible cascade and allows for the synthesis of large functions. \\*




\subsection{ESOP Cube List Ordering Rules}


Modifications to the above method %with the use of inverters instead of a negation line, careful ordering 
%of cubes to reduce the number of not gates and and sharing cubes between output lines 
have both reduced the number of lines and the number of gates required to implement these circuits. % see \emph{[To add citations for these papers]}.  
In \cite{Nayeem2011}, for example, a number of rules were proposed that manipulate the cube list representation to create a new cube list with 
the same output as the first one, 
depending on the rule and the state of the cube that the rule is being applied to these rules may increase 
or decrease the number of cubes in the list.\\


%\noindent \emph{[To add brief description of ordering rules as these are to be used in our mutation function ...]}

\newcommand{\tab}{\hspace*{2em}}

\subsection{Genetic Algorithm}
 The Genetic Algorithm is a search heuristic that was introduced and investigated by John Holland (1975) 
and his students (DeJong, 1975). The algorithm attempts to mimic the evolutionary process of natural 
selection (Mitchell, 1996) by modelling the concepts of individuals in a population, fitness and selection, 
crossover, and mutation that are found in the biological reproduction of organisms\cite{Mitchell1996}. 

The basic idea is that over a number of generations an individual that is the solution to a problem can 
evolve out of the latest generation of the individuals in the population. While there are a many variations 
on this theme, the basic algorithm is as follows:
\begin{enumerate}
 \item Generate an initial population of 'individuals' that represent potential solutions to the problem.
 \item Evaluate each individual's Fitness
 \item Until a solution is found or a maximum number of generations have passed, repeat the following:
  \begin{enumerate}
   \item Select individuals to reproduce.
   \item Apply crossover
   \item Apply mutation
   \item Evaluate each individual's fitness.
  \end{enumerate}

\end{enumerate}


\subsubsection{Representation}

Representation of the individuals in a population is critical to the successful application of this algorithm and is 
perhaps the most challenging aspect of implementing it. Since the model is based on biology, each individual is often 
encoded as some type of string which the mutation and crossover functions can be applied to. 

To show how a the genetic algorithm works we will use a simple example of adding two numbers to reach a specified sum. 
Each individual in our algorithm will be represented as bit strings consisting of 6 'genes' with each gene composed of 4
bits each. For example, an individual in our example may be represented as:  0001 1110 1001 1111 0110 0000. We choose an
encoding scheme where each gene in the individual encodes for either a number or an operator. We'll let  the genes 0000 to 
1101 represent the numbers 0 through 13 and the gene 1110 and 1111 can represent addition and subtraction respectively. Using 
this encoding scheme, our individual would represent the arithmetic expression 1 + 9 - 70.

\subsubsection{Fitness and Selection}
The fitness function allows us to measure how close the solution represented but any specific individual is to the desired 
output solution. Each individual is evaluated and ranked according to its fitness, and then according to the selection 
criteria some are chosen for crossover and mutation. To measure the fitness we need to decode the representation of each 
individual and compar it against our goal result. From the example individual above, if our goal was to find an expression 
whose result is equal to the absolute value of the number 30, our fitness function would decode the individual and find that 
it has a value of -60 and then compare it to the goal and give it a fitness, say 0.5. Once every individual is ranked according 
to its fitness, a number of individuals are selected for crossover and mutation according to the selection criteria that we specify.


\subsubsection{Crossover}
Crossover tries to model the process of sexual reproduction in nature. After parent individuals are selected from the 
population of the previous generation, the 'genes' of these individuals are combined to make one or more child individuals. To 
combine the genes a crossover point is specified in each parent individual, it is important that the crossover point occur between genes
and not inside of a gene or our encoding would be broken. Consider the following two binary strings, A: 1010 1100 0001 1000 1111 0000 
and B: 1011 0011 1010 1111 0000 1110. The offspring individuals will be a combinateion of the first part of A with the second part of B, 
or the first part of A with the second part of B. Here is an illustration of crossover using randomly chosen crossover points X:

\begin{center}
A: 1010 1100 0001 X 1000 1111 0000
\\*B: 1011 0011 1010 X 1111 0000 1110
\vspace{2 mm}
\\*We now have fragments 
\\*A1: 1010 1100 0001 
\\*A2: 1000 1111 0000
\\*B1: 1011 0011 1010 
\\*B2: 1111 0000 1110
\vspace{2 mm}
\\*Combining the fragments of A and B we would produce two offspring A1B2 and B1A2:
\vspace{2 mm}
\\*A1B2: 1010 1100 0001 1111 0000 1110
\\*B1A2: 1011 0011 1010 1000 1111 0000
\end{center}

\subsubsection{Mutation}
The mutation operation models mutation of the genes in organisms. For selected individuals individual 'genes' are randomly 
changed in some way (depending on your algorithm and encoding this mutation may be through addition, deletion, swapping of 
bits, or switching one gene another type of gene).
For example given the individual A:  1010 1100 0001 1000 1111 0000 mutation could occur through swapping a bit to resulting 
individual: A\(_{sw}\):   1010 1000 0001 1000 1111 0000, or by replacing a gene with another gene with the resulting individual:
A\(_{sw}\):   1010 1100 1110 1000 1111 0000.

\subsubsection{Initial Parameter Variations}

There are a number of other factors in the design of genetic algorithm that can impact the chances of finding a useful result. 
These factors include:
\begin{itemize}
 \item The initial population size.
\\ If you have too few individuals in your initial popluation you may not be able to search enough of the search space to 
find an acceptable solution, conversly, if you have too many you may be searchhing more of the search space than you need to.
 \item The maximum number of generations.
\\ Specifying a maximum number of generations ensures that your search will not continue indefinitely, however if this number is too 
small it limits the possibility for your algorithm to find a good solution.
 \item The fitness threshold that you are willing to accept.
\\ Depending on the application you may be able to accept a solution that is close-to but does not exacly provide the desired result, 
whereas in other situations you can only accept a solutiong that is guaranteed to give you the desired result. A fitness threshold 
allows you to specify how exact of a solution you need. 
\end{itemize}
 

\subsubsection{Applications of the Genetic Algorithm to Reversible Logic Synthesis}

There have been a number of applications og the genetic algorithm to logic synthesis in general, and reversible logic synthesis in 
particular. \cite{Lukac2003} \cite{Lukac2008} \cite{Khan2004} \cite{Aguirre2003}


%\emph{This section will detail our approach to the genetic algorithm, 
%including the planned use of the cube reordering rules in our mutation function
%and a general description of our sofware and the hardware it runs on}

\section{Experimental Results}

\emph{This section will detail our results}

\section{Conclusion}
We have presented a method to optimize reversible logic synthesis using genetic algorithms. Additionally, we have detailed the implementation of a reversible logic synthesis framework which allowed us to implement the genetic algorithm method to a large extent. \\

Our algorithm shows promise, but it is by no means optimal: further investigation is required in order to ascertain its effectiveness due to discrepancies in quantum cost calculation between our framework and those presented in RevLib. \\

Future research in this area should address topics such as function-preserving mutation and crossover functions, as these were points of difficulty in our implementation. As our research focused more on the parallel implementation of a genetic algorithm, some more time should be spent on finding optimal cascade representations as well as on overall scalability. Currently, due to the nature of our approach, we require that a cascade's truth table be calculated before any quantum cost evaluation can take place. In practice, it would be far more efficient to utilize function-preserving transformations which would eliminate this requirement. \\

In conclusion, even though our algorithm delivered sub-optimal results, we were able to develop an effective and solid framework for future research on the optimization of large reversible cascades.

\pagebreak
\bibliographystyle{plainnat}
\bibliography{bibliography}

\end{document}
