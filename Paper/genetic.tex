\newcommand{\tab}{\hspace*{2em}}

\subsection{Genetic Algorithm}
 The Genetic Algorithm is a search heuristic that was introduced and investigated by John Holland (1975) and his students (DeJong, 1975). The algorithm attempts to mimic the evolutionary process of natural selection by modelling the concepts of individuals in a population, fitness and selection, crossover, and mutation that are found in the biological reproduction of organisms (\cite{Mitchell1996}). \\

Over a number of generations, fit individuals (effective solutions to a problem) evolve out of the previous generations of individuals in a population. While there exist many variations on this theme, the basic algorithm is as follows:
\begin{enumerate}
 \item Generate an initial population of individuals that represent potential solutions to the problem, typically in a randomized fashion.
 \item Evaluate the ``fitness'' of each individual in the population according to some metric (a fitness function).
 \item Until a solution is found or a maximum number of generations have elapsed, repeat the following process:
  \begin{enumerate}
   \item Select individuals from the population to reproduce. This is typically done according to the heuristic provided by the fitness function.
   \item Perform crossover between individuals in order to create a new population.
   \item Apply mutations to certain members of the population in order to expand the search space.
   \item Evaluate each individual's fitness and keep track of the top individuals since they may be candidates for the next selection.
  \end{enumerate}
\end{enumerate}


\subsubsection{Representation}
Representation of the individuals in a population is critical to the successful application of this algorithm and is perhaps the most challenging aspect of implementing it. Since the model is based on biology, each individual is often 
encoded as a string to which the mutation and crossover functions are applied. \\

To show how the genetic algorithm works, we will use a simple example of adding two numbers to reach a specified sum. Each individual in our algorithm will be represented as bit strings consisting of 6 ``genes'' with each gene composed of 4
bits. Thus, an individual in our example may be represented as: 0001 1110 1001 1111 0110 0000. We choose an
encoding scheme where each gene in the individual encodes either a number or an operator. We will let the genes $0000$ to 
$1101$ represent the numbers 0 through 13 and the genes $1110$ and $1111$ represent addition and subtraction operators, respectively. Using 
this encoding scheme, our individual would represent the arithmetic expression $1 + 9 - 70$.

\subsubsection{Fitness and Selection}
The fitness function allows us to measure how close the solution represented by any individual is to the desired output solution. Each individual is evaluated and ranked according to its fitness, and then according to some selection 
criteria where candidates are chosen for crossover and mutation. To measure the fitness, we need to decode the representation of each 
individual and compare it against our goal result. From the example individual above, if our goal is to find an expression 
whose result is equal to the absolute value of the number $30$, our fitness function would decode the individual's genetic representation and find that 
it has a value of $-60$. The individual would then be compared to the goal and its fitness would be evaluated, yielding a value between $0.0$ and $1.0$, inclusive. Once every individual is ranked according 
to its fitness, candidates are selected for crossover and mutation according to the selection criteria specified.


\subsubsection{Crossover}
Crossover attempts to model the process of sexual reproduction found in nature. After ``parent'' individuals are selected from the population of a generation, the genes of these individuals are combined in order to produce one or more child individuals. To 
combine their genes, a crossover point is specified in each parent. Consider the following two binary strings, $A$ = 1010 1100 0001 1000 1111 0000 
and $B$ = 1011 0011 1010 1111 0000 1110. Their offspring individuals might be a combination of the first part of $A$ with the second part of $B$, 
or the first part of $A$ combined with the second part of $B$. Here is an illustration of crossover using a randomly chosen crossover point, $X$:

\begin{center}
$A$ = 1010 1100 0001 X 1000 1111 0000
\\*$B$ = 1011 0011 1010 X 1111 0000 1110
\vspace{2 mm}
\\*We now have the following fragments: 
\\*$A_{1}$ = 1010 1100 0001 
\\*$A_{2}$ = 1000 1111 0000
\\*$B_{1}$ = 1011 0011 1010 
\\*$B_{2}$ = 1111 0000 1110
\vspace{2 mm}
\\*Combining the fragments of $A$ and $B$ we would yield two offspring, $A_{1}B_{2}$ and $B_{1}A_{2}$:
\vspace{2 mm}
\\*$A_{1}B_{2}$ = 1010 1100 0001 1111 0000 1110
\\*$B_{1}A_{2}$ = 1011 0011 1010 1000 1111 0000
\end{center}

\subsubsection{Mutation}
The mutation operation models the natural mutation of genes in living organisms. For selected individuals, individual genes may be randomly 
changed in some way. Depending on algorithm and encoding, this mutation is usually achived through addition, deletion, or swapping of bits.
For example given the individual $A$ = 1010 1100 0001 1000 1111 0000, mutation could occur by swapping a bit. An individual resulting from this operation might be: 
\begin{center} $A_{sw}$ = 1010 1\underline{0}00 0001 1000 1111 0000 \end{center}
If instead we replace a gene with another gene within the individual, we might obtain:
\begin{center} $A_{re}$ =   1010 \underline{1100} 1110 1000 1111 0000 \end{center}
                                                           
\subsubsection{Initial Parameter Variations}

There are a number of other factors in the design of genetic algorithms that can impact the likelihood of finding a useful (or convergent, in some cases) result. 
These factors include:
\begin{itemize}
 \item {\bf Initial population size:}
\\ Having less individuals in the initial popluation may not allow us to search enough of the search space to find an acceptable solution. Conversely, having too many may result in the algorithm searching more of the search space than necessary.
 \item {\bf Maximum number of generations:}
\\ Specifying a maximum number of generations ensures that search will not continue indefinitely. However, if this number is too small it may limit the algorithm's ability to find a reasonable solution.
 \item {\bf Fitness threshold:}
\\Implementing a fitness threshold allows us to specify a desired ``precision'' for the solution. Depending on the application, an acceptable solution could be one that is very close to the ideal solution but does not exactly provide the
desired result, whereas in other situations, an acceptable solution could be the unique solution that provides a desired result. 
\end{itemize}
 

\subsubsection{Applications of the Genetic Algorithm to Reversible Logic Synthesis}

There have been a number of applications of the genetic algorithm to logic synthesis in general, and reversible logic synthesis in 
particular. \cite{Lukac2003} used genetic algorithm for reversible circuits to generate near-optimal circuits, to which they subsequently applied 
optimization transformations to generate optimal circuits. \cite{Lukac2008} implemented symbolic synthesis within the genetic algorithm to reduce the complexity of the resulting circuits, 
\cite{Khan2004} described a new genetic algorithm-based synthesis method for ternary quantum circuits which reduced gate count in some instances, and 
\cite{Aguirre2003} proposed the use of information theory as the basis for designing a fitness function for Boolean circuit design using genetic programming.