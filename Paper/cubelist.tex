\subsection{ESOP cube-list Representations}

Any Boolean function can be represented by an exclusive-or sum-of-products (ESOP) expression. This is particularly useful for reversible logic synthesis since there are existing algorithms for converting any ESOP expression into a cascade of Toffoli gates, thus allowing us to generate a reversible circuit from arbitrary Boolean functions. \\ 


In reversible circuit design, ESOP expressions are often written as a cube list. A cube list is an $ n \times m $ matrix, where $m$ is the number of product terms in the ESOP expression, and $n = i + j$, where $i$ is the number of input variables and $j$ is the number of output variables in the expression. Each of the $m$ rows in the matrix are the ``cubes'' that make up the cube list and represent one of the products from the ESOP expression. \\

Each cube in the list takes the general form: $x_{1} x_{2} ... x_{i} f_{1} f_{2} ... f_{j}$, where each of the elements $x_{1} ... x_{i}$ 
represent an input variable and each element $f_{1} ... f_{j}$ represents an output variable. For each cube in the 
cube list, a $1$ is written in cube position $x_{k}$ , $k \in \{1,2, ..., i\}$ if the variable $x_{k}$ is in 
the ESOP product for that row. A $0$ is written if the negation $\bar{x}_{k}$ is present, and a '$-$' is written if $x_{k}$ is not present in the
product term for that cube. For each element $f_{p}$, $p \in \{1,2,...j\}$, a 1 is written if that output variable contains 
the product represented by the input portion of the list and a 0 is written otherwise. See Figure \ref{fig:cubelist}a for 
an example. \\
%\include figures (a) and (b) showing cube-list and circuit.
\begin{figure}[h]
\centering
 \begin{tabular}{l l l}
  \begin{tabular}{lll lll}
    $x_{1}$ & $x_{2}$ & $x_{3}$ & $f_{1}$ & $f_{2}$ & $f_{3}$ \\
    \hline
    1 & 1 & 1 & 1 & 1 & 1 \\
    1 & 0 & 1 & 1 & 1 & 0 \\
    0 & 1 & $-$ & 1 & 1 & 1 \\
  \end{tabular} 
  & \ \ \ &
  \begin{tabular}{l}
  \Qcircuit @C=1.5em @R=1.0em {
    \lstick{x_{1}} 	 &  \ctrl{1}      &  \ctrl{1}      &  \ctrl{1}      &  \ctrl{1}      &  \ctrl{1}      &  \qw        &  \qw        &  \qw        &  \qw & \lstick{g_{1}} \\
    \lstick{\bar{x}_{1}} &  \qw           &  \qw           &  \qw           &  \qw           &  \qw           &  \ctrl{1}   &  \ctrl{1}   &  \ctrl{1}   &  \qw & \lstick{g_{2}} \\
    \lstick{x_{2}} 	 &  \ctrl{1} \qwx &  \ctrl{1} \qwx &  \ctrl{1} \qwx &  \qw \qwx      &  \qw \qwx      &  \ctrl{1}   &  \ctrl{1}   &  \ctrl{1}   &  \qw & \lstick{g_{3}} \\
    \lstick{\bar{x}_{2}} &  \qw           &  \qw           &  \qw           &  \ctrl{1} \qwx &  \ctrl{1} \qwx &  \qw        &  \qw        &  \qw        &  \qw & \lstick{g_{4}} \\
    \lstick{x_{3}} 	 &  \ctrl{1} \qwx &  \ctrl{1} \qwx &  \ctrl{1} \qwx &  \ctrl{1}      &  \ctrl{1}      &  \qw \qwx   &  \qw \qwx   &  \qw \qwx   &  \qw & \lstick{g_{5}} \\
    \lstick{\bar{x}_{3}} &  \qw           &  \qw           &  \qw           &  \qw           &  \qw           &  \qw \qwx   &  \qw \qwx   &  \qw \qwx   &  \qw & \lstick{g_{6}} \\
    \lstick{0}  	 &  \targ  \qwx   &  \qw \qwx      &  \qw \qwx      &  \targ  \qwx   &  \qw \qwx      &  \targ \qwx &  \qw \qwx   &  \qw \qwx   &  \qw & \lstick{f_{1}} \\
    \lstick{0}  	 &  \qw           &  \targ \qwx    &  \qw \qwx      &  \qw           &  \targ  \qwx   &  \qw        &  \targ \qwx &  \qw \qwx   &  \qw & \lstick{f_{2}} \\
    \lstick{0}   	 &  \qw           &  \qw           &  \targ \qwx    &  \qw           &  \qw           &  \qw        &  \qw        &  \targ \qwx &  \qw & \lstick{f_{3}} 
    }
  \end{tabular} \\
  \ \ \ \ \ \ \ \ \ \ \ \ (a) & \ \ \ & \ \ \ \ \ \ \ \ \ \ \ \ \ \ \ \ \ \ \ \ \ \ \ \ \ \ \ \ \ \ (b)
 \end{tabular}
 \caption{(a) The cube list and (b) resulting circuit.}
  \label{fig:cubelist}
\end{figure}
 
Given an ESOP expression encoded as a cube list, \cite{Thornton2007} proposed a 
method for reversible logic synthesis that implements the function as a reversible circuit using only Toffoli gates. In this method, an empty cascade with $2i + j$ lines is created. Two input lines are given for every input 
variable where one line corresponds to the variable $x_{k}$, the next line corresponds to its negation, $\bar{x}_{k}$. The remaining $j$ lines correspond to the output variables. For every output line $f_{p}$, a Toffoli gate 
is placed with its target on line $f_{p}$ and a control for is placed on the input line $x_{k}$ if there is a 1 in the cube for the corresponding variable, or if we encounter a 0 in the
cube for $x_{k}$, we place the control on the negation line of $x_{k}$. see Figure \ref{fig:cubelist}b. This method allows a cube list to be efficiently transformed into a reversible cascade and allows for the synthesis of large functions. \\*


