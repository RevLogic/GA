\subsection{Parallelization}

Parallelization is an approach to computational problem solving where the computation is divided into smaller 
subproblems and each sub-problem is computed simultaneously. The results of each sub-problem are then combined 
to get the final result of the whole computation.

The process of parallelizing a computation can be taken at different levels, from the bit level on a single machine 
to distributed computing over multiple machines (such as cluster or grid computing).

\paragraph{multiple threading and processes}
A common approach is to split a computation into into multiple threads or processes where it can be concurrently 
worked on by different processors on the same cpu (references)

\paragraph{Grid Computing}
add basic definition: which is each node is generally loosely coupled, generally heterogeneous.